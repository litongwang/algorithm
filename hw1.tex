\documentclass[12pt,letterpaper]{article}
\usepackage{amsmath, amssymb}
\usepackage{fullpage}
\pagestyle{empty}
\def\pp{\par\noindent}

%%%%%%%%%%%%%%%%%%%%%%%%%%%%%%%%%%%%%%%%%%%%%%%%%%%%%%%%%%%%%%%%%%%%%%%%%%%%%%

\renewcommand{\baselinestretch}{1.2}
\newcommand{\problem}[1]{ \bigskip \pp \textbf{Problem #1}\par}
\newcommand{\solution}{\textit{Solution:}\par}
\newcommand{\answer}{\medskip\pp\textit{Answer:} }
\newcommand{\lemma}[1]{\medskip\pp\textit{Lemma #1:}}
\newcommand{\proof}{\medskip\pp\textit{Proof:} }
\newcommand{\hint}[1] {\par{\footnotesize {\bf Hint:} #1}}
\newcommand{\remark}[1]{\par{\footnotesize {\bf Remark:} #1}}

%%%%%%%%%%%%%%%%%%%%%%%%%%%%%%%%%%%%%%%%%%%%%%%%%%%%%%%%%%%%%%%%%%%%%%%%%%%%%%

\newcommand{\bbZ}    {\mathbb{Z}}
\newcommand{\bbQ}    {\mathbb{Q}}
\newcommand{\bbN}    {\mathbb{N}}
\newcommand{\bbB}    {\mathbb{B}}
\newcommand{\bbR}    {\mathbb{R}}
\newcommand{\bbC}    {\mathbb{C}}
\newcommand{\calP}   {{\cal{P}}}

%%%%%%%%%%%%%%%%%%%%%%%%%%%%%%%%%%%%%%%%%%%%%%%%%%%%%%%%%%%%%%%%%%%%%%%%%%%%%%
\DeclareMathOperator{\E}{E}
\DeclareMathOperator{\Var}{Var}
\DeclareMathOperator{\cov}{cov}
%%%%%%%%%%%%%%%%%%%%%%%%%%%%%%%%%%%%%%%%%%%%%%%%%%%%%%%%%%%%%%%%%%%%%%%%%%%%%%

\begin{document}

\centerline{\bf EECS 336}

\medskip
\centerline{Litong Wang}
\centerline{Homework 1}
\centerline{Due on April 11, 2016}
\bigskip


%%%%%%%%%%%%%%%%%%%%%%%%%%%%%%%%%%%%%%%%%%%%%%%%%%%%%%%%%%%%%%%%%%%%%%%%%%%%%%
\begin{enumerate}

\item % Problem 1: 

Modify Problem 3-3 in the textbook by squaring every function in columns 1, 3, and 5. Solve this problem for the functions in the second and third rows. Justify your answers.\\

The 12 functions are:\\
$$(\frac{3}{2})^{2n},n^3,\lg ^4n,\lg (n!),2^{2^{n+1}},n^{\frac{1}{\lg n}},(\ln\ln n)^2,\lg ^* n,n^2 \cdot 2^{2n},n^{\lg \lg n},\ln^2 n,1$$

According to the text book, I can get the following conclusions: \\
$$ n! = \Omega(a^n), a^n = \Omega(n^k) (a > 1, k > 1), n^k = \Omega (n) (k > 1), n = \Omega(\lg n), \lg n =\Omega (\lg ^* n), \lg ^* n =\Omega(1)$$
a. The following functions are ranked by order of growth: 
$$2^{2^{n+1}},n^2 \cdot 2^{2n},(\frac{3}{2})^{2n},n^{\lg \lg n},n^3,\lg (n!),\lg ^4n,\ln^2 n,(\ln\ln n)^2,\lg ^* n, n^{\frac{1}{\lg n}} , 1$$

$$2^{2^{n+1}}= \Omega(n^2 \cdot 2^{2n})$$
Prove: \\
Method 1):
\begin{align*}
n^2 \cdot 2^{2n} &= 2 ^ {\lg (n^2 \cdot 2^{2n})} \\
&= 2 ^ {2n + 2 \lg n} \\
c &= 1, n_0 = 2, \\
\forall n &\ge n_0, 2^{2^{n+1}} \ge n^2 \cdot 2^{2n} \\
\therefore 2^{2^{n+1}} &= \Omega(n^2 \cdot 2^{2n})
\end{align*}
Method 2):
\begin{align*}
& \lim_{x \to \infty} \frac{\lg 2 ^ { 2 ^ {n+1}}}{\lg n ^ 2 \cdot 2 ^{2n}} = \lim_{x \to \infty} \frac{2^{n+1}}{2 \lg n + 2n} = \infty \\
& \therefore \lim_{x \to \infty} \frac{2^{2^{n+1}}}{n^2 \cdot 2^{2n}} = \infty \therefore 2^{2^{n+1}} = \omega(n^2 \cdot 2^{2n}) \therefore 2^{2^{n+1}} = \Omega(n^2 \cdot 2^{2n})
\end{align*}

$$n^2 \cdot 2^{2n} = \Omega ((\frac{3}{2})^{2n})$$
Prove: \\
Method 1):
\begin{align*}
n^2 \cdot 2^{2n} &= 2 ^ {2n + 2 \lg n} \\
c &= 1, n_0 = 1, \\
\because \forall n& \ge n_0, 2 ^ {2n + 2 \lg n} \ge \frac{3}{2} ^ {2n} \\
\therefore \forall n& \ge n_0, n^2 \cdot 2^{2n} \ge \frac{3}{2} ^ {2n} \\
\therefore n^2 \cdot 2^{2n} &= \Omega ((\frac{3}{2})^{2n})
\end{align*}
Method 2):
\begin{align*}
& \lim_{n \to \infty} \frac{n^{2n} \cdot 2^{2n}}{(\frac{3}{2})^{2n}} = \infty \\
& \therefore n^2 \cdot 2^{2n} = \omega((\frac{3}{2})^{2n}) \therefore n^2 \cdot 2^{2n} = \Omega((\frac{3}{2})^{2n})
\end{align*}

$$(\frac{3}{2})^{2n} = \Omega (n^{\lg \lg n})$$
Prove: \\
Method 1):
\begin{align*}
(\frac{3}{2})^{2n} &= 2 ^ {2n \cdot \lg \frac{3}{2}} \\
n ^ {\lg \lg n} &= 2 ^ {\lg n \cdot \lg \lg n} \\
c &= 1, n_0 = 2, \\
\because \forall n& \ge n_0, 2n \cdot \lg \frac{3}{2} \ge \lg n \cdot \lg \lg n \\
\therefore \forall n& \ge n_0, (\frac{3}{2})^{2n} \ge n ^ {\lg \lg n} \\
\therefore (\frac{3}{2})^{2n} &= \Omega (n^{\lg \lg n})
\end{align*}
Method 2):
\begin{align*}
& \lim_{n \to \infty} \frac{\lg ((\frac{3}{2})^{2n})}{\lg (n^{\lg \lg n})} = \lim_{n \to \infty} \frac{2n \lg (\frac{3}{2})}{\lg \lg n \cdot \lg n} = \infty \\
& \therefore \lim_{n \to \infty} \frac{(\frac{3}{2})^{2n}}{n ^ {\lg \lg n}} = \infty \therefore (\frac{3}{2})^{2n} = \omega(n^{\lg \lg n}) \therefore (\frac{3}{2})^{2n} = \Omega (n ^ {\lg \lg n})
\end{align*}

$$n^{\lg \lg n} = \Omega(n^3)$$
Prove: \\
Method 1):
\begin{align*}
c &= 1, n_0 = 2 ^ 8 \\
\because \forall n & \ge n_0, \lg \lg n \ge 3 \\
\therefore \forall n & \ge n_0, n^{\lg \lg n} \ge n ^ 3 \\
\therefore n^{\lg \lg n} & = \Omega(n^3)
\end{align*}
Method 2):
\begin{align*}
& \lim_{n \to \infty} \frac{n^{\lg \lg n}}{n^3} = \lim_{n \to \infty} \frac{\lg \lg n \cdot \lg n}{3 \lg n} = \infty \\
& \therefore \lim_{n \to \infty} \frac{n ^ {\lg \lg n}}{n^3} = \infty \therefore n^{\lg \lg n} = \omega (n^3) \therefore n^{\lg \lg n} = \Omega(n^3)
\end{align*}

$$n^3 = \Omega (\lg (n!))$$
Prove: \\
Method 1):
\begin{align*}
\lg (n!) &= \Theta(n \lg n) \\
\because n^2 &= \Omega(\lg n) \\
\therefore n^3 &=\Omega(n \lg n) \\
\therefore n^3 &= \Omega(\lg (n!))
\end{align*}
Method 2):
\begin{align*}
& \lim_{n \to \infty} \frac{n^3}{\lg (n!)} = \lim_{n \to \infty} \frac{n^3}{n \lg n} = \lim_{n \to \infty} \frac{n^2}{\lg n} = \infty \\
& \therefore \lim_{n \to \infty} \frac{n^3}{\lg (n!)} = \infty \therefore n^3 = \omega(\lg (n!)) \therefore n^3 = \Omega(\lg (n!))
\end{align*}

$$\lg (n!) = \Omega (\lg ^4 n)$$
Prove: \\
Method 1):
\begin{align*}
\lg (n!) &= \Theta(n \lg n) \\
c &= 1, n_0 = 2 ^ 10, \\
\because \forall n& \ge n_0, n \lg n \ge \lg ^4 n \\
\therefore \lg (n!) &= \Omega (\lg ^4 n)
\end{align*}
Method 2):
\begin{align*}
& \lim_{n \to \infty} \frac{n \lg n}{\lg ^4 n} = \lim_{n \to \infty} \frac{n}{\lg ^ 3 n} = \infty \\
& \therefore \lg (n!) = \omega(\lg ^ 4 n) \therefore \lg (n!) = \Omega(\lg ^ 4 n)
\end{align*}

$$\lg ^4 n = \Omega(\ln ^2 n)$$
Prove: \\
Method 1):
\begin{align*}
\because 2 &< 4 \\
\therefore \lg ^4 n &= \Omega(\ln ^2 n)
\end{align*}
Method 2):
\begin{align*}
& \lim_{n \to \infty} \frac{\lg ^4 n}{\ln ^2 n} = \infty \\
&\therefore \lg ^ 4 n = \omega (\ln ^ 2 n) \therefore \lg ^ 4 n = \Omega(\ln ^2 n)
\end{align*}

$$\ln ^2 n = \Omega((\ln \ln n)^2)$$
Prove: \\
Method 1):
\begin{align*}
c &= 1, n_0 = e \\
\because \forall n & \ge n_0, n \ge \ln n \\
\therefore \forall n & \ge n_0, \ln n \ge \ln \ln n \\
\therefore \forall n & \ge n_0, \ln ^2 n \ge (\ln \ln n)^2 \\
\therefore \ln ^2 n &= \Omega((\ln \ln n)^2)
\end{align*}
Method 2):
\begin{align*}
& \lim_{n \to \infty} \frac{(\ln ^2 n)}{(\ln \ln n)^2} = \infty \\
& \therefore \ln ^ 2 n = \omega((\ln \ln n)^2) \therefore \ln ^ 2 n = \Omega((\ln \ln n)^2)
\end{align*}

$$(\ln \ln n)^2 = \Omega(\lg ^* n)$$
Prove: \\
Method 1):
\begin{align*}
\lg^*(2) &= 1 \\
\lg^*(4) &= 2 \\
\lg^*(16) &= 3 \\
\lg^*(65536) &= 4 \\
\lg^*(2^{65536}) &= 5 \\
c = 1, n_0 & = 16 \\
\because \forall n \ge n_0, &(\ln \ln n)^2 \ge \lg ^* n \\
\therefore (\ln \ln n)^2 &= \Omega(\lg ^* n)
\end{align*}
note that $2^{65536}$ is much larger than the number of atoms in the observable universe. Function $\lg ^* n$ grows so slow that it could pretty much be considered as constant time. \\
Method 2):
\begin{align*}
& \lim_{n \to \infty} \frac{(\ln \ln n)^2}{\lg ^* n} = \infty \\
& \therefore (\ln \ln n)^2 = \omega (\lg ^* n) \therefore (\ln \ln n)^2 = \Omega(\lg ^* n)
\end{align*}

$$\lg ^* n =\Omega(n^{\frac{1}{\lg n}})$$
Prove: \\
Method 1):
\begin{align*}
n^{\frac{1}{\lg n}} &= x \\
\frac{1}{\lg n} \cdot \lg n &= \lg x \\
x &= 2 \\
n^{\frac{1}{\lg n}} & = 2 \\
when \quad n &\ge 4 \\
\lg ^* n &\ge 2 \\
\therefore \lg ^* n &=\Omega(n^{\frac{1}{\lg n}})
\end{align*}
Method 2):
\begin{align*}
& \lim_{n \to \infty} \frac{\lg ^* n}{n ^{\frac{1}{\lg n}}} = \lim_{n \to \infty} \frac{\lg ^* n}{2} = \infty \\
& \therefore \lg ^* n = \omega(n^{\frac{1}{\lg n}}) \therefore \lg ^* n = \Omega(n^{\frac{1}{\lg n}})
\end{align*}

$$n^{\frac{1}{\lg n}} = \Theta(1)$$
Prove:
\begin{align*}
n^{\frac{1}{\lg n}} &= 2 \\
\therefore n^{\frac{1}{\lg n}} &= \Theta(1)
\end{align*}

b. Give an example of a single nonnegative function $f(n)$ such that for all functions $g_i(n)$ in part (a), $f(n)$ is neither $\mathcal{O}(g_i(n))$ nor $\Omega(g_i(n))$.\\

Function $f(n) = 2 ^ {2 ^ {(n+1) \sin n}}$ is neither $\mathcal{O}(g_i(n))$ nor $\Omega(g_i(n))$ for all functions $g_i(n)$. \\ 

For any $c$, there exist $n = \frac{(2k+1)\pi}{2}$ such that $f(n) > c g(n)$ \\

For any $c$, there exist $n = \frac{2k \pi}{2}$ such that $f(n) < c g(n)$ \\

Therefore, $ f(n) $ is neither $\mathcal{O}(g(n))$ or $\Omega(g(n))$

\item % Problem 2
In the following, the functions F1(n), F2(n), ..., F6(n) and G1(n), G2(n), ..., G6(n) refer to the functions in the first and fourth rows, respectively, in Problem 3-3 in the textbook.\\

Simplify the following six functions as much as possible in terms of the asymptotic Theta notation. Justify your answers. \\

In order to simplify the functions as much as possilble in terms of the asymptotic Theta notation. We need to find the term which grows the fastest, and the Theta notation of that term will be our answers. \\

$1) \quad F_1(n) + G_1(n) \times G_2(n) = \lg (\lg ^* n) + 2^{\lg n} \times (\lg n)^{\lg n} $ \\

The term which grows the fastest is the second term: $ 2^{\lg n} \times (\lg n) ^ {\lg n}$ \\

So the Theta notation of this function should be $\Theta( 2^{\lg n} \times (\lg n) ^ {\lg n} ) = \Theta((2 \lg n)^{\lg n})$ \\

$ 2) \quad F_2(n) \times \log(G_2(n)) + (G_4(n))^3 = 2^{\lg ^* n} \times \log((\lg n)^{\lg n}) + (4^{\lg n})^3 $ \\

The term which grows the fastest is the second term: $(4^{\lg n})^3$ \\

So the Theta notation of this function should be $ \Theta((4^{\lg n})^3) = \Theta(n^6) $ \\

$ 3) \quad F_3(n)+G_3(n)+ \log(G_6(n)) = (\sqrt{2})^{\lg n} + e^n + \log(\sqrt{\lg n}) $ \\

The term which grows the fastest is the second term: $ e^n$ \\

So the Theta notation of this function should be $\Theta(e^n) $ \\

$ 4) \quad F_4(n) \times (G_4(n))^3 + F_3(n) = n^2 \times (4^{\lg n})^3 + (\sqrt{2})^{\lg n} $ \\

The term which grows the fastest is the first term: $ n^2 \times (4^{\lg n})^3 $ \\

So the Theta notation of this function should be $\Theta(n^2 \times (4^{\lg n})^3 ) = \Theta(n^8) $ \\

$ 5) \quad F_5(n) \times F_6(n) \times G_6(n) + F_2(n) = n! \times (\lg n)! \times \sqrt{\lg n} + 2 ^ {\lg ^* n} $\\

The term which grows the fastest is the first term: $ n! \times (\lg n)! \times \sqrt{\lg n} $ \\

So the Theta notation of this function should be $\Theta(n! \times (\lg n)! \times \sqrt{\lg n}) $ \\

$ 6) \quad 2^{G_5(n)}+F_3(n) = 2^{(n+1)!} + (\sqrt{2})^{\lg n} $ \\

The term which grows the fastest is the first term: $ 2^{(n+1)!} $ \\

So the Theta notation of this function should be $\Theta(2^{(n+1)!}) $ \\

\item % Problem 3
Problem 3-5 in the textbook.\\
Some author define $\Omega$ in a slightly different way than we do; let's use $\stackrel{ \infty} \Omega$ (read “omega infinity”) for this alternative definition. We say that $f(n)=\stackrel{ \infty} \Omega (g(n))$ if there exists a positive constant c such that $f(n) \ge c g(n) \ge 0$ for infinitely many integers n.\\

a. Show that for any two functions $f(n)$ and $g(n)$ that are asymptotically nonnegative, either $f(n)=\mathcal{O}(g(n))$ or $f(n)=\stackrel{ \infty} \Omega (g(n))$ or both, whereas this is not true if we use $\Omega$ in place of $\stackrel{ \infty} \Omega $.\\

Prove: for any two functions $f(n)$ and $g(n)$, we can have the following piecewise function:\\
\begin{equation}
   f(n)=
   \begin{cases}
   \mathcal{O}(g(n)) \ \& \  \stackrel{ \infty} \Omega (g(n)) &\mbox{if $f(n)=\Theta(g(n))$}\\
   \mathcal{O}(g(n)) &\mbox{if $0 \le f(n) \le c g(n)$}\\
   \stackrel{ \infty} \Omega (g(n)) &\mbox{if $0 \le c g(n) \le f(n)$ for infinitely many integers n}\\
   \end{cases}
  \end{equation}
Hence we can say that $f(n)=\mathcal{O}(g(n))$ or $f(n)=\stackrel{ \infty} \Omega (g(n))$ or both are true for both two functions.\\

However, it is not true for $\Omega$, for example, $n^{\sin n} = \stackrel{\infty} {\Omega} (n)$ but $n^{\sin n} \neq \Omega (n)$.

b. Describe the potential advantages and disadvantages of using $\Omega$ instead of $\stackrel{ \infty } { \Omega }$ to characterize the running times of programs.\\

Advantages: By using $\stackrel{\infty} {\Omega}$ we can use Asymptotic notation to present more relation between two functions.\\

Disadvantages: The relation presented by $\stackrel{\infty} {\Omega}$ is not as accurate as the one presented using $\Omega$. \\

Some authors also define $\mathcal{O}$ in a slightly different manner; let’s use $\mathcal{O}'$ for the alternative definition. We say that $|f(n)| = \mathcal{O}' (g(n))$ if and only if $f(n) = \mathcal{O}' (g(n))$.\\

c. What happens to each direction of the “if and only if” in Theorem 3.1 if we substitute $\mathcal{O}'$ for $\mathcal{O}$ but still use $\Omega$? \\

Theorem 3.1: For any two functions $f(n)$ and $g(n)$, we have $f(n) = \Theta(g(n))$ if and only if $f(n) = \mathcal{O}(g(n))$ and $f(n) = \Omega(g(n))$.\\

If we subsitute $\mathcal{O}'(g(n))$ for $\mathcal{O}(g(n))$, we can have $f(n) = \Theta(g(n))$ only if $f(n) = \mathcal{O}'(g(n))$ and $f(n) = \Omega(g(n)).$ Which gives us: \\

$ \quad f(n) = \Theta(g(n)) \Rightarrow f(n) = \mathcal{O}'(g(n)) $ and $f(n) = \Omega (g(n))$. \\
Prove:
\begin{align*}
& \because f(n) = \Theta(g(n)), \\
& \therefore \exists c_1, c_2, n_0, \forall n \ge n_0, 0 \le c_1 g(n) \le f(n) \le c_2 g(n) \\
& \therefore \exists c,n_0, \forall n \ge n_0, 0 \le c g(n) \le f(n), \\
& \therefore f(n) = \Omega(g(n)), \\
& \therefore \exists c,n_0, \forall n \ge n_0,  0 \le f(n) \le c g(n), \\
& \therefore f(n) = \mathcal{O}(g(n)), \\
& \therefore f(n) = \mathcal{O}'(g(n)), \\
& \therefore f(n) = \Theta(g(n)) \Rightarrow f(n) = \mathcal{O}'(g(n)) \ \& \ f(n) = \Omega(g(n))
\end{align*}

According to the definition of $\Omega$ in the textbook we have $0 \le f(n)$, then: \\

$ \quad f(n) = \mathcal{O}'(g(n))$ and $f(n) = \Omega (g(n)) \Rightarrow f(n) = \Theta(g(n)).$ \\
Prove:
\begin{align*}
& \because f(n) = \Omega (g(n)), \\
& \therefore \exists c_1, n_0, \forall n \ge n_0, 0 \le c_1 g(n) \le f(n) \\
& \because f(n) = \mathcal{O}'(g(n)), \\
& \therefore \exists c_2, n_0, \forall n \ge n_0, |f(n)| \le c_2 g(n), \\
& \because f(n) \ge 0, f(n) \le |f(n)|, \\
& \therefore \exists c_1, c_2, n_0, \forall n \ge n_0, 0 \le c_1 g(n) \le f(n) \le c_2 g(n), \\
& \therefore f(n) = \Theta(g(n))
\end{align*}

Therefore: we can prove that $f(n) = \Theta(g(n)) \Leftrightarrow f(n) = \Omega(g(n)) \ \& \ f(n) = \mathcal{O}(g(n)) $\\

However, if we use the other definition of $\Omega$ which cannot give us $0 \le f(n)$, then we cannot prove that: \\

$f(n) = \mathcal{O}'(g(n)) \ \& \ f(n) = \Omega(g(n)) \Rightarrow f(n) = \Theta(g(n)) $ \\

Some authors define $\stackrel{\sim}{\mathcal{O}}$ (read "soft-oh") to mean $\mathcal{O}$ with logarithmic factor ignored:\\

$\stackrel{\sim}{\mathcal{O}}(g(n)) = \{ f(n):$ there exist positive constant c, k, and $n_0$ such that $0 \le f(n) \le c g(n) \lg ^k (n) $ for all $n \ge n_0 \}.$\\

d. Define $\stackrel{\sim}{\Omega}$ and $\stackrel{\sim}{\Theta}$ in a similar manner. Prove the corresponding analog to Theorem 3.1.\\

$\stackrel{\sim}{\Omega}(g(n)) = \{ f(n):$ there exist positive constant c, k, and $n_0$ such that $0 \le c g(n) \lg^{-k} n \le f(n) $ for all $n \ge n_0 \}$ \\

$f(n) = \Omega(g(n)), f(n) = \mathcal{O}(g(n)) \Leftrightarrow f(n) = \Theta(g(n))$ \\

$\Rightarrow:$ \\

According to the definition, there exist positive constant $c_1, c_2, k_1, k_2,$ and $n_1, n_2$ such that $0 \le c_1 g(n) \lg ^{-k_1} (n) \le  f(n)$, for all $n \ge n_1$ and $0 \le f(n) \le c_2 g(n) \lg ^{k_2} (n)$, for all $n \ge n_2$. \\

Set $n_0 = max(n_1, n_2)$, then we can have exist positive constant $c_1, c_2, k_1, k_2,$ and $n_0$ such that $0 \le c_1 g(n) \lg ^{-k_1} (n) \le  f(n) \le c_2 g(n) \lg ^{k_2} (n)$, for all $n \ge n_0$. $\Rightarrow f(n) = \Theta(g(n))$ \\

$\Leftarrow:$ \\

According to the definition, there exist positive constant $c_1, c_2, k_1, k_2, n_0$ such that for all $n \ge n_0, 0 \le c_1 g(n) \lg ^{-k_1} (n) \le f(n) \le c_2 g(n) \lg ^{k_2} (n)$. \\

Therefore, we can have for all $n \ge n_0$, $ 0 \le c_1 g(n) \lg ^{-k_1} (n) \le f(n) $ and $ 0 \le f(n) \le c_2 g(n) \lg ^{k_2} (n)$. Therefore, we can have $f(n) = \Omega(g(n))$ and $f(n) = \mathcal{O}(g(n))$. \\

$\stackrel{\sim}{\Theta}(g(n)) = \{ f(n):$ there exist positive constant $c_1,c_2, k_1,k_2$ and $n_0$ such that $0 \le c_1 g(n) \lg^{-k_1} (n) \le f(n) \le c_2 g(n) \lg ^{k_2} (n) $ for all $n \ge n_0 \}$\\

\item % Problem 4
Modify Problem 3-6 in the textbook by adding 2 to each value of c in the table. Solve parts a, c, e, g, h. Prove your answers.\\

We can apply the iteration operator $^*$ used in the $\lg^*$ function to any monotonically increasing function $f(n)$ over the reals. For a given constant $c \in \mathbb{R}$, we define the iterated function $f_c^*$ by \\
$f_c*(n) = min \{ i \ge 0 : f^(i)(n) \le c\},$
which need not be well defined in all cases. In other words, the quantity $f_c^*(n)$ is the number of iterated applications of the function f required to reduce its argument down to c or less. \\
For each of the following functions $f(n)$ and constants c, give as tight a bound as possible on $f_c^*(n)$.\\

\begin{table}
\begin{tabular}[t]{|c|c|c|}
\hline
$f(n)$ & c & $f_c^*(n)$\\
\hline
$n-1$ & 2 & $\Theta(n) $\\
\hline
$n/2$ & 3 & $\Theta(\lg n )$\\
\hline
$\sqrt{n}$ & 4 & $\Theta (\lg \lg n)$\\
\hline
$n ^ \frac{1}{3}$ & 4 & $\Theta (\lg \lg n)$\\
\hline
$\frac{n}{\lg n}$ & 4 & $o(\lg n) \  \omega(\lg \lg n)$\\
\hline
\end{tabular}
\end{table}

$f_c^*(n) = min \{ i \ge 0: f^{(i)}(n) \le c \}$ \\

For $f(n) = n - 1 , c = 2$, $f_c^*(n) = \Theta(n)$
\begin{align*}
f_c^*(n) &= min \{ i \ge 0: f^{(i)}(n) \le 2 \} \\
&= min \{ i \ge 0: n - i \le 2 \} \\
&= n - 2 \\
\therefore f_c^*(n) &= \Theta(n)
\end{align*}

For $f(n) = n / 2 , c = 3 $, $f_c^*(n) = \Theta(\lg n)$
\begin{align*}
f_c^*(n) &= min \{ i \ge 0: f^{(i)}(n) \le 3 \} \\
&= min \{ i \ge 0: \frac{n}{2^i} \le 3 \} \\
&= min \{ i \ge 0:  \frac{n}{3} \le 2^i \} \\
&= min \{ i \ge 0: \lg n - \lg 3 \le i \} \\
&= \lg n - \lg 3 \\
\therefore f_c^*(n) &= \Theta(\lg n)
\end{align*}

For $f(n) = \sqrt{n} , c = 4 $, $f_c^*(n) = \Theta(\lg \lg n)$
\begin{align*}
f_c^*(n) &= min \{ i \ge 0: f^{(i)}(n) \le 4 \} \\
&= min \{ i \ge 0: n ^ {\frac{1}{2^i}} \le 4 \} \\
&= min \{ i \ge 0: \frac{1}{2^i} \lg n \le \lg 4 \} \\
&= min \{ i \ge 0: \frac{\lg n}{2} \le 2^i \} \\
&= min \{ i \ge 0: \lg \lg n - 1 \le i \} \\
&= \lg \lg n - 1 \\
\therefore f_c^*(n) &= \Theta(\lg \lg n)
\end{align*}

For $f(n) = n ^{\frac{1}{3}}, c = 4 $, $f_c^*(n) = \Theta(\lg \lg n) $
\begin{align*}
f_c^*(n) &= min \{ i \ge 0: f^{(i)}(n) \le 4 \} \\
&= min \{ i \ge 0: n ^ {\frac{1}{3^i}} \le 4 \} \\
&= min \{ i \ge 0: \frac{1}{3^i} \lg n \le \lg 4 \} \\
&= min \{ i \ge 0: \frac{\lg n}{2} \le 3^i \} \\
&= min \{ i \ge 0: \log_3 \lg n - \log_3 2 \le i \} \\
&= \log_3 \lg n - \log_3 2 \\
\therefore f_c^*(n) &= \Theta(\lg \lg n)
\end{align*}

For $ f(n) = \frac{n}{\lg n}, c = 4 $, $f_c^*(n) = \omega(\lg \lg n), f_c^*(n) = o(\lg n)$
\begin{align*}
\because \sqrt{n} = \frac{n}{\sqrt{n}} &\le \frac{n}{\lg n} \le \frac{n}{2} \\
According \ to \ & \ c) \ and \ e) \\
\therefore  \lg \lg n - 1 \le f_c^*&  \le \lg n - 2 \\
\therefore f_c^*(n) = \omega(\lg \lg n), & f_c^*(n) = o(\lg n)
\end{align*}

\end{enumerate}
\end{document}
