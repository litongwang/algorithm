\documentclass[12pt,letterpaper]{article}
\usepackage{amsmath, amssymb,algorithmic}
\usepackage[]{algorithm2e}
\usepackage{fullpage}
\usepackage{mathtools}
\DeclarePairedDelimiter{\ceil}{\lceil}{\rceil}
\DeclarePairedDelimiter\floor{\lfloor}{\rfloor}
\pagestyle{empty}
\def\pp{\par\noindent}

%%%%%%%%%%%%%%%%%%%%%%%%%%%%%%%%%%%%%%%%%%%%%%%%%%%%%%%%%%%%%%%%%%%%%%%%%%%%%%

\renewcommand{\baselinestretch}{1.2}
\newcommand{\problem}[1]{ \bigskip \pp \textbf{Problem #1}\par}
\newcommand{\solution}{\textit{Solution:}\par}
\newcommand{\answer}{\medskip\pp\textit{Answer:} }
\newcommand{\lemma}[1]{\medskip\pp\textit{Lemma #1:}}
\newcommand{\proof}{\medskip\pp\textit{Proof:} }
\newcommand{\hint}[1] {\par{\footnotesize {\bf Hint:} #1}}
\newcommand{\remark}[1]{\par{\footnotesize {\bf Remark:} #1}}

%%%%%%%%%%%%%%%%%%%%%%%%%%%%%%%%%%%%%%%%%%%%%%%%%%%%%%%%%%%%%%%%%%%%%%%%%%%%%%

\newcommand{\bbZ}    {\mathbb{Z}}
\newcommand{\bbQ}    {\mathbb{Q}}
\newcommand{\bbN}    {\mathbb{N}}
\newcommand{\bbB}    {\mathbb{B}}
\newcommand{\bbR}    {\mathbb{R}}
\newcommand{\bbC}    {\mathbb{C}}
\newcommand{\calP}   {{\cal{P}}}

%%%%%%%%%%%%%%%%%%%%%%%%%%%%%%%%%%%%%%%%%%%%%%%%%%%%%%%%%%%%%%%%%%%%%%%%%%%%%%
\DeclareMathOperator{\E}{E}
\DeclareMathOperator{\Var}{Var}
\DeclareMathOperator{\cov}{cov}
%%%%%%%%%%%%%%%%%%%%%%%%%%%%%%%%%%%%%%%%%%%%%%%%%%%%%%%%%%%%%%%%%%%%%%%%%%%%%%

\begin{document}

\centerline{\bf EECS 336}

\medskip
\centerline{Litong Wang}
\centerline{Homework 3}
\centerline{April 25, 2016}
\bigskip


%%%%%%%%%%%%%%%%%%%%%%%%%%%%%%%%%%%%%%%%%%%%%%%%%%%%%%%%%%%%%%%%%%%%%%%%%%%%%%
\problem{1}
Describe the generalization of the FFT procedure to the case in which n is a power of 5. Give a recurrence for the running time, and solve the recurrence. \\
\solution
Partition $A(x)$ to define five new polynomials $A^{[0]}(x)$, $A^{[1]}(x)$ , $A^{[2]}(x)$, $A^{[3]}(x)$ and $A^{[4]}(x)$ of degree-bound $\frac{n}{5}$:
\begin{align*}
& A^{[0]}(x) = a_0 + a_5 x + \cdots + a_{n-5} x^{\frac{n}{5}-1} \\
& A^{[1]}(x) = a_1 + a_6 x + \cdots + a_{n-4} x^{\frac{n}{5}-1} \\
& A^{[2]}(x) = a_2 + a_7 x + \cdots + a_{n-3} x^{\frac{n}{5}-1} \\
& A^{[3]}(x) = a_3 + a_8 x + \cdots + a_{n-2} x^{\frac{n}{5}-1} \\
& A^{[4]}(x) = a_4 + a_9 x + \cdots + a_{n-1} x^{\frac{n}{5}-1} \\
& A(x) = A^{[0]}(x^5) + x A^{[1]}(x^5) + x^2 A^{[2]}(x^5) + x^3 A^{[3]}(x^5) + x^4 A^{[4]}(x^5)
\end{align*}
Recursively evaluate $ A^{[0]}(x),A^{[1]}(x),A^{[2]}(x),A^{[3]}(x),A^{[4]}(x)$ at the $\frac{n}{5}$-th roots of unity: \\
$v^0,v^1,\cdots,v^{\frac{n}{5}-1}, v^k = (\omega^5)^k$. \\
Evaluate A(x) at the n-th roots of unity.
\begin{align*}
& A(\omega^k) = A^{[0]}(v^k) + \omega^{k} A^{[1]}(v^k) + \omega^{2k} A^{[2]}(v^k) +\omega^{3k} A^{[3]}(v^k) + \omega^{4k} A^{[4]}(v^k) \\
& A(\omega^{k+\frac{n}{5}}) =  A^{[0]}(v^k) + \omega^{k+\frac{n}{5}} A^{[1]}(v^k) + \omega^{2(k+\frac{n}{5})} A^{[2]}(v^k) +\omega^{3(k+\frac{n}{5})} A^{[3]}(v^k) + \omega^{4(k+\frac{n}{5})} A^{[4]}(v^k) \\
& A(\omega^{k+\frac{2n}{5}}) = A^{[0]}(v^k) + \omega^{k+\frac{2n}{5}} A^{[1]}(v^k) + \omega^{2(k+\frac{2n}{5})} A^{[2]}(v^k) +\omega^{3(k+\frac{2n}{5})} A^{[3]}(v^k) + \omega^{4(k+\frac{2n}{5})} A^{[4]}(v^k) \\
& A(\omega^{k+\frac{3n}{5}}) = A^{[0]}(v^k) + \omega^{k+\frac{3n}{5}} A^{[1]}(v^k) + \omega^{2(k+\frac{3n}{5})} A^{[2]}(v^k) +\omega^{3(k+\frac{3n}{5})} A^{[3]}(v^k) + \omega^{4(k+\frac{3n}{5})} A^{[4]}(v^k) \\
& A(\omega^{k+\frac{4n}{5}}) = A^{[0]}(v^k) + \omega^{k+\frac{4n}{5}} A^{[1]}(v^k) + \omega^{2(k+\frac{4n}{5})} A^{[2]}(v^k) +\omega^{3(k+\frac{4n}{5})} A^{[3]}(v^k) + \omega^{4(k+\frac{4n}{5})} A^{[4]}(v^k) \\
& 0 \le k < \frac{n}{5} 
\end{align*}
Runtime complexity:
\begin{align*}
T(n) &= 5T(\frac{n}{5}) + \Theta(n) \\
&= \Theta(n \lg n).
\end{align*}

\problem{2}
Consider three sets X, Y and Z, each having n integers in the range from 0 to 10n. We wish to compute the Cartesian sum of X, Y and Z, defined by
$ C = \{ x + y + z : x \in X, y \in Y, z \in Z\}$
Note that the integers in C are in the range from 0 to 30n. We want to find the elements of C and the number of times each element of C is realized as a sum of elements in X, Y and Z. Show how to solve the problem in $\mathcal{O}(n \lg n)$ time. (Hint: Represent X, Y and Z as polynomials of degree at most 10n.) \\
\solution
We can define $ X(x) = \sum\limits_{i=0}^n x^{a_i} , Y(x) = \sum\limits_{i=0}^n x^{b_i} \ and\  Z(x) = \sum\limits_{i=0}^n x^{c_i}\ where\  a_i \in X, b_i \in Y, c_i \in Z$.
We can compute $C(x) = X(x) \cdot Y(x) \cdot Z(x)$ using FFT in $\mathcal{O}(n \lg n)$ time.
\begin{align*}
C(x) &= X(x) \cdot Y(x) \cdot Z(x) \\
&= \sum\limits_{i=0} ^{n} d_i x^{a_i+b_i+c_i} \\
\end{align*}
We can see that $C_i = (a_i+b_i+c_i) \in C, C = \{ x + y + z : x \in X, y \in Y, z \in Z\}$ and $d_i$ is the number of times of $C_i$ appearing in $C$.

\problem{3}
Derive a point-value representation for $A^{rev}(x) = \sum\limits_{j=0}^{n-1}a_{n-1-j}x^{j}$ from a point-value representation for $A(x) = \sum\limits_{j=0}^{n-1}a_{j}x^{j}$ assuming that none of the points is 0. \\
\solution
\begin{align*}
\frac{A(x)}{x^{n-1}} &= \sum\limits_{j=0}^{n-1} a_j \frac{x^j}{x^{n-1}} \\
&= \sum\limits_{j=0}^{n-1} a_j \frac{1}{x^{n-1-j}} \\
&= \sum\limits_{i=0}^{n-1} a_{n-1-i} \frac{1}{x^i} \\
&= A^{rev}(\frac{1}{x}) \\
\end{align*}
$\therefore$ we can get a point-value representation for $A^{rev}(x)$ $\{x_i',y_i'\}$ from a point-value representation for $A(x)$ such that $ x_i' = x_i, y_i' = \frac{y_i}{x_i^{n-1}}$ \\

\problem{4}
Show that the solution to $T(n) = 2T(\floor[\big]{\frac{n}{2}}+ \ceil[\big]{\log n}) + n$ is $\Theta(n \lg n)$ \\
\solution
\begin{align*}
& g(n) = \floor[\big]{\frac{n}{2}} \\
& \frac{n}{2} - 1 < g(n) \le \frac{n}{2} \\ 
& \therefore |g(n)| \in \mathcal{O}(n^c) \\
& h(n) = \ceil[\big]{\log n} \\
& \log n \le h(n) < \log n +1 \\
& \because \lim_{n \to \infty} \frac{\lg (\log n)^3 }{\lg n} = \lim_{n \to \infty} \frac{\log n}{\frac{n}{(\log n)^2}} = \lim_{n \to \infty} \frac{3 \lg \log n}{\lg n } = 0 \\
& \therefore h(n) \in |\frac{n}{(\log n)^2}| \\
& \because 0 < a, 0 < b < 1 \\
& We\ can\ apply\ Akra-Bazzi\ Master\ Theorem \\
& \because P=1, a(b)^P = 1 \\
& T(n) = \Theta(n(1+\int^n_1 \, \frac{u}{u^2} du )) \\
& T(n) = \Theta(n(1+\int^n_1 \, d \log u )) \\
& T(n) = \Theta(n \log n) 
\end{align*}

\problem{5}
What is the largest k such that if you can multiply $5 \times 5$ matrices using k multiplications (not assuming commutativity of multiplication), then you can multiply $ n \times n$ matrices in time $o(n^{\lg 7})$? What would the running time of this algorithm be? \\
\solution
We can multiply $n \times n$ matrices by multiplying $\frac{n}{5} \times \frac{n}{5}$ matrices recursively. Hence, we have: $T(n) = k T(\frac{n}{5}) + \Theta(n^2)$. \\
Apply master method: \\
\begin{align*}
& a = k, b = \frac{1}{5}, \alpha = 2, \beta = 0 \\
& a (b)^x = 1 \\
& if\  x < \alpha, case\ 1: T(n) = \Theta(n^2) \\
& T(n) = o(n^{\lg 7}) \\
& \therefore a \cdot \frac{1}{25} < 1 \\
& \therefore k = a < 25 \\
& if\  x = \alpha, case\ 2: T(n) = \Theta(n^2 (\log n)) \\
& T(n) = o(n^{\lg 7}) \\
& \therefore a \cdot \frac{1}{25} = 1 \\
& \therefore k = a = 25 \\
& if\  x > \alpha, case\ 3: T(n) = \Theta(n^x) \\
& \because T(n) = o(n^{\lg 7}) \\
& \therefore x < \lg 7 \\
& \because a = \frac{1}{b^x} \\
& \therefore a \le 5^{\lg 7} \\
\end{align*}
The largest integer such that smaller than $ 5^{\lg 7} $ is 91 \\
The running time is $\Theta(n^{\log_5 91})$ \\

\problem{6}
Assume you have an array X[1..n] of n elements. A majority element of X is defined to be an element occurring in more than $\frac{n}{2}$ positions (e.g., if $ n = 6 $ or $ n = 7$, a majority element will occur in at least 4 positions). Assume that elements cannot be ordered or sorted, but can be compared for equality. (You might think of the elements as chips, and there is a tester that can be used to determine whether or not two chips are identical.) \\
a) Design an algorithm to find a majority element in X or determine that no majority element exists. The time complexity of your algorithm should be $\mathcal{O}(n \log n)$. \\
b) Design an algorithm to find a majority element in X or determine that no majority element exists. The time complexity of your algorithm should be $\mathcal{O}(n)$. \\
Prove the correctness and time complexity of each of your algorithms. \\
\solution
a) Split the array A into two arrays $A_1,A_,2$ with half size of A, we look for the majority elements in both arrays. If the two majority elements in two subarrays are same, then we can just return this element. If they are not, we need to compute the frequency of the majority elements. This operation takes $\mathcal{O}(n)$ time. Then we can decide which is the majority element or the majority element doesn't exist. \\
Algorithm: Majority element1 \\
\begin{algorithm}[H]
\KwData{$A[1,\cdots,n]$}
\KwResult{The majority element}
\If{$n=1$} {
	$return A[1]$ \;
}
$ mid = \frac{n}{2}$\;
$ left = Majority\  element1(A[1,mid])$\;
$ right = Majority\  element1(A[mid+1,n])$\;
\eIf{ $left = right$ }{
	$return\  left$ \;
	}{
	$f_{left} = getFrequency(left,A[1,n])$ \;	
	$f_{right} = getFrequency(right,A[1,n])$ \;
	}
 \If{$ f_{left} > \frac{n}{2} $}{
   $return\  left$ \;
   }
\If{$ f_{right} > \frac{n}{2} $}{
   $return\  right$ \;
   }
   $return\  None$ \;
\end{algorithm}
Correctness: \\
Proof. \\
1) If A has a majority element $e$, then $e$ must be the majority element of $A_1\ or\ A_2\ or\ both$. \\
2) Base case: $n=1$ return A[1], the algorithm is correct. \\
3) Induction hypothesis: \\
In step k, assume that for input size is n. If A has a majority element $e$, then $e$ must be the majority element of $A_1\ or\ A_2\ or\ both$. \\
In next step, the input size is 2n, we have three scenarios: \\
Majority element $A_1$ of $A$ and majority element $B_1$ of $B$. \\
a) $A_1 = B_1$, $frequency(A_1) > \frac{n}{2}\ and\  frequency(B_1)> \frac{n}{2}$ then $frequency(A_1)+frequency(B_1) > n$, we can get the majority element. \\
b) $A_1 \neq B_1$, then according to the algorithm, we should check the $frequency(A_1)\ and\ frequency(B_1)$. If frequency of each element is greater than n, then we find the majority element or there is no majority element. \\
c) $A_1\ and\ B_1$ don't exist, then there is no majority element.
4) We reduce the search range by half in each iteration, then the algorthm will always terminate. \\
Time complexity:
\begin{align*}
T(n) &= 2T(\frac{n}{2}) + \mathcal{O}(n) \\
&= \mathcal{O}(n \log n )
\end{align*}
\solution
b) We use a hashtable to store pair $(element,frequency)$. We go through the whole array, if element is already in hashtable, we update the frequency$(frequency++)$ or we insert a new pair $(element,frequency),frequency = 1$ into the hashtable. We also create a variable count to store and update the highest frequency. If after looking through the whole array, $count \le \frac{n}{2}$, then there is no majority element. \\
Algorithm: Majority element2 \\
\begin{algorithm}[H]
\KwData{$A[1,\cdots,n]$}
\KwResult{The majority element}
\If{$n=1$} {
	$return A[1]$ \;
}
Create a hashtable: hash \;
count = 0 \;
\While{$ i \le n $}{
	\eIf{$ hash.contains(A[i]) $}{
   $hash.put(A[i],hash.get(A[i]+1))$ \;
   $count = max(count,hash.get(A[i]+1))$ \;
   }{
	$hash.put(A[i],1)$ \;   
   }
   $i++$ \;
   \If{$ count > \frac{n}{2} $}{
   $return\  A[i]$ \;
   }
}
 $return\  None$ \;
\end{algorithm}
Correctness: \\
We keep track of the frequency of each element. If the majority element exist, by definition, its $frequency > \frac{n}{2}$. If there is no element which has frequency higher than $\frac{n}{2}$, then the majority element doesn't exist. \\
Time complexity: \\
Because the runtime of update the hashtable is $\mathcal{O}(1)$ and we only need to do n iteration to search through the whole array. So the time complexity of this algorithm is $\mathcal{O}(n)$.
\end{document}
