\documentclass[12pt,letterpaper]{article}
\usepackage{amsmath, amssymb,algorithmic}
\usepackage[]{algorithm2e}
\usepackage{fullpage}
\usepackage{mathtools}
\DeclarePairedDelimiter{\ceil}{\lceil}{\rceil}
\DeclarePairedDelimiter\floor{\lfloor}{\rfloor}
\pagestyle{empty}
\def\pp{\par\noindent}

%%%%%%%%%%%%%%%%%%%%%%%%%%%%%%%%%%%%%%%%%%%%%%%%%%%%%%%%%%%%%%%%%%%%%%%%%%%%%%

\renewcommand{\baselinestretch}{1.2}
\newcommand{\problem}[1]{ \bigskip \pp \textbf{Problem #1}\par}
\newcommand{\solution}{\textit{Solution:}\par}
\newcommand{\answer}{\medskip\pp\textit{Answer:} }
\newcommand{\lemma}[1]{\medskip\pp\textit{Lemma #1:}}
\newcommand{\proof}{\medskip\pp\textit{Proof:} }
\newcommand{\hint}[1] {\par{\footnotesize {\bf Hint:} #1}}
\newcommand{\remark}[1]{\par{\footnotesize {\bf Remark:} #1}}

%%%%%%%%%%%%%%%%%%%%%%%%%%%%%%%%%%%%%%%%%%%%%%%%%%%%%%%%%%%%%%%%%%%%%%%%%%%%%%

\newcommand{\bbZ}    {\mathbb{Z}}
\newcommand{\bbQ}    {\mathbb{Q}}
\newcommand{\bbN}    {\mathbb{N}}
\newcommand{\bbB}    {\mathbb{B}}
\newcommand{\bbR}    {\mathbb{R}}
\newcommand{\bbC}    {\mathbb{C}}
\newcommand{\calP}   {{\cal{P}}}

%%%%%%%%%%%%%%%%%%%%%%%%%%%%%%%%%%%%%%%%%%%%%%%%%%%%%%%%%%%%%%%%%%%%%%%%%%%%%%
\DeclareMathOperator{\E}{E}
\DeclareMathOperator{\Var}{Var}
\DeclareMathOperator{\cov}{cov}
%%%%%%%%%%%%%%%%%%%%%%%%%%%%%%%%%%%%%%%%%%%%%%%%%%%%%%%%%%%%%%%%%%%%%%%%%%%%%%

\begin{document}

\centerline{\bf EECS 336}

\medskip
\centerline{Litong Wang}
\centerline{Homework 8}
\centerline{May 30, 2016}
\bigskip


%%%%%%%%%%%%%%%%%%%%%%%%%%%%%%%%%%%%%%%%%%%%%%%%%%%%%%%%%%%%%%%%%%%%%%%%%%%%%%
\problem{1}
\solution
We need to show that a verification algorithm $V(x,y)$ can verify
language $GRAPH-ISOMORPHISM = \{ \left \langle G_1, G_2 \right \rangle : G_1\ and\ G_2\ are\ isomorphic\ graphs\}$
in polynomial time. \\
We let $x$ to be the two graph $G_1$ and $G_2$ and let the certificate $y$ to be a map $\{ f: v \in A, f(v) \in B \}$ from nodes in $G_1$ to nodes in $G_2$.
The verification algorithm runs as follows: \\
Withour loss of generality, \\
1) Check whether every node in $G_1$ is in $A$ and every node in $G_2$ is in $B$. If no, return false; else continue. \\
2) Check whether every node in $A$ is also in $G_1$ and check every node in $B$ is also in $G_2$. If no, return false; else continue. \\
3) Check whether there is an edge$(u,v)$ in $G_1$ from $u$ to $v$ iff there is an edge$(f(u),f(v))$ in $G_2$ from $f(u)$ to $f(v)$. If yes, return true; else return false. \\
Time complexity: \\
The first two steps both takes $\mathcal{O}(|V|^2)$ time to run and the third step takes $\mathcal{O}(|E|)$ time to run.
So this alorithm runs in polynomial time. \\
$\because$ The problem $GRAPH-ISOMORPHISM$ can be verified in polynomial time. \\
$\therefore$ $GRAPH-ISOMORPHISM \in NP$.

\problem{2}
\solution
If we want to show that A decision problem is a member of $co-NP$, we need to prove its complement is in the complexity class NP.
We define the complement of $TAUTOLOGY$ to be $co-TAUTOLOGY$: \\
If the boolean formulas are tautologies then output "NO",
if one of the boolean formulas is not tautology then ouput "YES". \\
One of the inputs $x$ to $V$ is a boolean combinational circuit $C$.
The other input $y$ is a certificate corresponding to an assignment of boolean values to the wires in $C$.
We construct the algorithm $V$ as follows.
We assign the value of input of $C$ the same as the value in $y$.
For each logic gate in the circuit $C$, we correctly compute the output determined on inputs.
If the output of the entire $C$ is 0, then the algorithm output 1.
Since the values assigned to the inputs of $C$ proves that it is not a tautoglogy.
Otherwise, $V$ outputs 0. \\
Whenever a circuit which is not tautology is input to algorithm $V$,
there exists a certificate whose length is polynomial in the size of $C$ and cause the output of $C$ to be 0.
Whenever a circuit is a tautology, no certificate can make $C$ output 0.
Hence, no certificate can fool $V$ into believing $C$ is not tautology. \\
Assume the input size of $C$ is $k$, the runtime of this algorithm is $\mathcal{O}(k)$.
Thus, we can verify $co-TAUTOLOGY$ in polynomial time, and $co-TAUTOLOGY \in NP$. Therefore, $TAUTOLOGY \in co-NP$

\problem{3}
\solution
a) Algorithm for 2-coloring problem: \\
\begin{algorithm}[H]
\KwData{$G=(V,E)$ and two color red and blue}
\KwResult{coloring of $G$}
Set $Neighbours = \emptyset$ \;
Set $Result = \emptyset$ \;
$Neighbours = \{ v \}, v$ is any arbitray node in $V$\;
$V = V- v$ \;
$v.color = red$ \;
\While{$Neighbours$ is not empty}{
  pick one $v$ such that $v \in Neighbours$ \;
  add the neighbours of $v$ into $Neighbours$ \;
  remove these neighbours from $V$ \;
   \eIf{$v.color == red$}{
    \For{Each vertex $u \in v.neighbours$ } {
      \eIf{$u.color == v.color$} { return $None$ \;}
      {$u.color = blue$ \;}
    }
   }{
   \For{Each vertex $u \in v.neighbours$ } {
    \eIf{$u.color == v.color$} { return $None$ \;}
     {$u.color = red$ \;}
   }
   }
   add $v$ into $Result$ \;
   connect corresponding edges \;
   remove $v$ from $Neighbours$ \;
}
return $Result$ \;
\end{algorithm}
Correctness: \\
If we find that the coloring is legal,
which means the two endpoints of any edge in $G$ has different colors,
then we determine a 2-coloring of a graph. \\
If we find that it is an illgeal coloring, then the graph cannot be legally colored by two color.
Claim: \\
Let $u$ be a vertex in $G$.
Any color that $u$ gets from the program is the color that $u$ must get if the $v$ is colored Red.
In base case, $u = v$ the claim is clearly correct.
If the distance between $u$ and $v$ is $d+1$, then $u$ must have a neghbour $u'$ such that has distance $d$ from $v$.
By the indcution hypothesis, the color of $u'$ is the color that if $v$ is colored Red.
Then $u$ only has one choice of color which is Red if $u'$ is Blue or Blue if $u'$ is Red.
Hence, the claim is correct.
We reach a contradiction because some vertex $u$ in $G$ is colored both Red and Blue by the program. \\
Time complexity: \\
We use bread-first-search alogrithm here.
We only do coloring to each vertex once, hence the runtime of this algorithm is $\mathcal{O}(V)$.
Hence, this algorithm is an efficient algorithm. \\
b) The decision problem $D$: Given a graph $G=(V,E)$ and an integer $k$, does a k-coloring exist? Output "Yes" if it exists or "No" if it doen't. \\
Assume we can solving graph-coloring problem in polynomial time and the solution uses $k'$ colors to do coloring of the graph.
Then we can solve the decision problem $D$ by checking if $k \ge k'$.
If $k \ge k'$ then it is an "YES" instance of decision problem.
If $k < k'$ then it is an "NO" instance of decision problem.
In contrast, assume we can solve the decision problem $D$ in polynomial time.
$\left \langle G, V, E, k \right \rangle$ is a "YES" instance of decision problem $D$ and we can solve it in $\mathcal{O}(n^c)$.
The graph-coloring problem can be solved by assigning $k$ from $1$ to $|V|$ and applying at most $|V|$ times of $D$.
That means the graph-coloring problem can be also solved in $\mathcal{O}(|V|n^c)$ time. \\
Hence,  that your decision problem $D$ is solvable in polynomial time if and only if the graph-coloring problem is solvable in polynomial time. \\
c)
We first construct a new decision problem $D'$, Given a graph $G=(V,E)$, does a 3-coloring exist? Output "Yes" if it exists or "No" if it doen't. \\
The proof consists of two parts: \\
Part 1. Decision problem $D \in NP$: \\
Let $x$ be the origin graph $G=(V,E)$ and the certificate $y$ be a graph $G'$ which is colored with $k$ colors.
The verfication algorithm $V$ runs as follows: \\
1) Check whether origin graph $G$ is identical to colored graph $G'$. If no, false; else continue. \\
2) Check whether the end points of every edge in $G$ have different colors. If no, false; else continue. \\
3) Check whether exactly $k$ colors are used in $G'$. If no, false; else yes. \\
The runtime of the first step is $\mathcal{O}(|V|+|E|)$, the runtime of the second step is $\mathcal{O}(|E|)$ and the runtime of the thrid step is $\mathcal{O}(|V|)$. \\
Hence, the decision problem $D$ can be verified in polynomial time. Hence, the decision problem $D \in NP$. \\
Part 2. $D' \le_{p} D$: \\
We need to present a polynomial-time many-one reduction from $D'$ to decision problem $D$. \\
In order to prove $D' \le_{p} D$, we need to find a polynomial-time algorithm such that can transform inputs to $D'$ into inputs to decision problem $D$,
such that the transformed problem has the same output as the original problem.
An instance $x$ of $D'$ problem can be solved by applying the reduction algorithm and we get an instance $y$ of decision problem $D$.
We can use the result of instance $y$ of decision problem $D$ to be the result of instance $x$ of $D'$ problem.
Let $x$ to be the graph $G$, and let $y$ to be the graph $G'$.
We can construct $G'$ by the following steps: \\
1) connect every vertex in origin graph $G$ to a vertex $v$, assign $v$ a new color. \\
2) keep doing the above procedure until we add $k-3$ vertices(add $k-3$ colors) to the origin graph $G$.
Then the current graph is $G'$, return $G'$. \\
Correctness: \\
If graph $G$ is 3-colorable and instance $\left \langle G, V, E \right \rangle$ of problem $D'$ is a "YES" instance,
then $\left \langle G', V, E, k \right \rangle$ must be a "YES" instance of decision problem $D$.
Because in each iteration we add one vertex into the origin graph, this vertex is the only vertex that has the additional color.
For the origin vertices in $G$, we color these vertices in $G'$ exactly like the color assign in $G$.
And every additional color is only assigned to one vertex.
If graph $G$ is 3-colorable, adding these vertices won't change the result. \\
If decision problem $D$ returns true.
Because each additional vertex in $G'$ is connected to every origin vertices in $G$,
then every additional vertex is the only vertex has its color.
Then if we remove all these additional vertices, the graph $G$ will only have 3 colors.
Removing all these additional vertices won't change the result of decision problem $D$.
Thus, the graph $G$ must be 3-colorable.
Time complexity: \\
Adding one vertex into the graph takes linear time.
We need to add $k-3$ vertices, so the construction of graph $G'$ takes polynomial time. \\
Hence, $D' \le_{p} D$.
According to b), decision problem $D'$ is solvable in polynomial time if and only if the 3-coloring problem is solvable in polynomial time.
Hence, $3-color \le_{p} D$ \\
Hence, the decision problem $D$ is $NP-complete$.

\problem{4}
\solution
A Hamiltonian cycle can be expressed as a permutation $Perm$ of $\{1,2,\cdots,n\}$
such that $Perm(p) = q$ means the $p$th position is occupied by vertex $q$.
And edge $(Perm(p),Perm(p+1)) \in G, 1 \le p \le n, Perm(n + 1) = Perm(1)$. \\
a) $HC \le_{p} CNF-SAT$. \\
Given a graph $G=(V,E)$, we should construct a boolean formula $R(G)$ such that $R(G)$ is satisfiable iff $G$ has a Hamiltonian cycle. \\
Let $v(1),v(2),\cdots,v(n)$ be the vertices of $G$.
Let $w(1),w(2),\cdots,w(n)$ be our target Hamiltonian cycle in $G$.
Convention: $x(p,q)=true$ iff $w(p) = v(q)$.
To describe that $w(1),w(2),\cdots,w(n)$ is a Hamiltonian cycle in $G$, we will say the following in terms of $CNF-SAT$ clauses: \\
$R(G)$ has $|V|^2$ boolean variables $x_{ij}$ such that $1 \le i \le j \le |V|$.
The boolean formula $R(G)$ should include the following clauses: \\
1) Every vertex in $G$ should be in the cycle. \\
Each $v(q)$ is at least one of $w(1), w(2), \cdots , w(n)$. \\
$(x(w(1),v(q)) \vee x(w(2),v(q)) \vee \cdots \vee x(w(n),v(q)))$ for $q, 1 \le q \le n$. \\
2) No vertex appear twice in the cycle. \\
Each $v(q)$ is at most one of $w(1), w(2), \cdots , w(n)$. \\
$(\neg x(w(p),v(q)) \vee \neg x(w(p'),v(q)))$ for every $p,p',q, 1 \le q \le n, 1 \le p \le n, 1\le p' \le n, p \neq p'$. \\
3) Every position $i$ in the cycle should be occupied. \\
Each $w(p)$ is at least one of $v(1),v(2),\cdots,v(n)$. \\
$(x(w(p),v(1)) \vee x(w(p),v(2)) \vee \cdots \vee x(w(p),v(n))$ for each $1 \le p \le n$. \\
4) No two vertices will occupy the same position $i$ in cycle. \\
Each $w(p)$ is at most one of $v(1), v(2), \cdots, v(n)$ \\
$(\neg x(w(p),v(q)) \vee \neg x(w(p),v(q')))$ for every $p,q,q' 1 \le q \le n, 1 \le q' \le n, 1\le p \le n, p \neq p'$. \\
5) If two vertices are not adjacent, they cannot be adjacent in graph $G$. \\
For each $p$, $w(p)$ and $w(p+1)$ form an edge in $G$, $w(n+1) = w(1)$. \\
$(\neg x(w(p),v(q)) \vee \neg x(w(p+1),v(q')))$ for all edge $(v(q),v(q')) \notin G, 1 \le p \le n $. \\
By this way, we can construct a $CNF-SAT$ problem. \\
Correctness: \\
If graph $G$ is a Hamiltonian cycle, then it fulfills all the five conditions above.
Then all clauses in five sets should return true.
Hence, the boolean formula is a "YES" instance of problem $CNF-SAT$. \\
If all clauses in five sets are true, the boolean formula is a "YES" instance of problem $CNF-SAT$.
The first and the second sets of clauses guarantee that $1 \le q \le n$
is only assign to the right of equation of exacly one $Perm$ in $Perm$ of $\{1,2,\cdots,n\}$,
which means every vertex appears in the cycle only once.
The third and the forth sets of clauses guarantee that each
$Perm(p), 1 \le p \le n$ exactly has one value which means every position in the cycle is occupied only once.
The firth set of clauses guarantee that $Perm$ of $\{1,2,\cdots,n\}$ is a cycle in $G$.
Hence, the $Perm$ of $\{1,2,\cdots,n\}$ is a Hamiltonian cycle in $G$. \\
Time complexity: \\
The first set of clauses consists of $\mathcal{O}(n)$ clauses, then it has $\mathcal{O}(n^2)$ literals.
The second set of clauses consists of $\mathcal{O}(n^2)$ clauses, then it has $\mathcal{O}(n^3)$ literals.
The third set of clauses consists of $\mathcal{O}(n)$ clauses, then it has $\mathcal{O}(n^2)$ literals.
The fourth set of clauses consists of $\mathcal{O}(n^2)$ clauses, then it has $\mathcal{O}(n^3)$ literals.
The firth set of clauses consists of $\mathcal{O}(n^2)$ clauses, then it has $\mathcal{O}(n^3)$ literals.
The runtime of constructing one clause is $\mathcal{O}(1)$, hence, the total runtime is $\mathcal{O}(n^3)$. \\
In conclusion, $G$ is a "YES" instance of $HC$ iff $R(G)$ is a "YES" instance of $CNF-SAT$.
The algorithm from $HC$ to $CNF-SAT$ runs in polynomial time.
Therefore, $HC \le_{p} CNF-SAT$.
b) $HC \le_{p} 01-LP$. \\
Given a graph $G=(V,E)$, we should construct a set of constraints $R(G)$ such that $R(G)$ is satisfiable iff $G$ has a Hamiltonian Cycle. \\
Let $v(1),v(2),\cdots,v(n)$ be the vertices of $G$.
Let $w(1),w(2),\cdots,w(n)$ be our target Hamiltonian cycle in $G$.
Convention: $x(p,q)=1$ iff $w(p) = v(q)$.
To describe that $w(1),w(2),\cdots,w(n)$ is a Hamiltonian cycle in $G$, we will say the following in terms of 01-LP problem: \\
1) Every vertex appear only once in the cycle. \\
Each $v(q)$ is exactly one of $w(1), w(2), \cdots, w(n)$. \\
$x(w(1),v(q)) + x(w(2),v(q)) + \cdots + x(w(n),v(q)) = 1$ for $q, 1 \le q \le n$. \\
2) Every position $i$ in the cycle should be occupied by only one vertex. \\
Each $w(p)$ is exactly one of $v(1), \cdots, v(n)$. \\
$x(w(p),v(1)) + x(w(p),v(2)) + \cdots + x(w(p),v(n)) = 1$ for $p, 1 \le p \le n$. \\
3) If two vertices are not adjacent, they cannot be adjacent in graph $G$. \\
For each $p$, $w(p)$ and $w(p+1)$ form an edge in $G$, $w(n+1) = w(1)$. \\
$x(w(p),v(q)) + x(w(p+1),v(q')) < 2$ for all edge $(v(q),v(q')) \notin G, 1 \le p \le n$. \\
Correctness: \\
If graph $G$ is a hamiltonian cycle, then it fulfills all the three conditions above.
Then all three sets of constraints should hold.
Hence, this is a "YES" instance of problem 01-LP. \\
If all constraints in three sets hold, the boolean formula is a "YES" instance of problem 01-LP.
The first set of constraints guarantee that $1 \le q \le n$
is only assign to the right of equation of exacly one $Perm$ in $Perm$ of $\{1,2,\cdots,n\}$,
which means every vertex appears in the cycle only once.
The second set of constraints guarantee that each
$Perm(p), 1 \le p \le n$ exactly has one value which means every position in the cycle is occupied only once.
The third set of constraints guarantee that $Perm$ of $\{1,2,\cdots,n\}$ is a cycle in $G$.
Hence, the $Perm$ of $\{1,2,\cdots,n\}$ is a Hamiltonian cycle in $G$. \\
Time complexity: \\
The first set of constraints consists of $\mathcal{O}(n)$ constraints, then it has $\mathcal{O}(n^2)$ variables.
The second set of constraints consists of $\mathcal{O}(n^2)$ constraints, then it has $\mathcal{O}(n^3)$ variables.
The third set of constraints consists of $\mathcal{O}(n^2)$ constraints, then it has $\mathcal{O}(n^3)$ variables.
The runtime of constructing one clause is $\mathcal{O}(1)$, hence, the total runtime is $\mathcal{O}(n^3)$. \\
In conclusion, $G$ is a "YES" instance of $HC$ iff $R(G)$ is a "YES" instance of $01-LP$.
The algorithm from $HC$ to $01-LP$ runs in polynomial time.
Therefore, $HC \le_{p} 01-LP$. \\
\end{document}
